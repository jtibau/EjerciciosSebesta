\subsubsection{Pregunta 4:Escribir en C\# un programa que pase por referencia un literal a un subprograma, que intenta cambiar el parametro. Dado la filosofia total de C\#, explique los resultados.}
\lstset{language = C}  %Seteo para poner Codigo de Lenguaje C
Código creado para la explicación en C\#.
\begin{lstlisting}[frame = single] %Comienzo del Código

 static void Main(string[] args)
        {
            String a;
            int b= -1;
            Program p = new Program();
            System.Console.Write("Ingrese numero:\n");
            a = Console.ReadLine();
            System.Console.Write("El numero es: " +a);
            System.Console.Write("\n");
            b = Convert.ToInt32(a);
            p.fun(ref b);
            System.Console.Write("El numero enviado a la funcion por referencia es: "+b);
            System.Console.Read();
            //System.Console.ReadKey();
        }

        public void fun (ref int x){
            x = x + 5;  
        }
\end{lstlisting}
OUTPUT:\\
	Ingrese numero: 6\\
	El numero es: 6\\
	El numero enviado a la funcion por referencia es:11\\
	El lenguague C\# recibe por defecto cuando se ingresa por teclado un Strign a este valor hay que convertirlo a un entero para enviarlo a nuestra funcion con la que queremos mostrar lo que deseamos.\\
 Lo que queremos mostrar es el cambio que se produce al valor de una variable siendo enviada a un subrprograma por referencia.\\

