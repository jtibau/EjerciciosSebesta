\subsubsection{C5 Pregunta6: Sentencia for alcance de la variabe declarada en ella en los tres lenguajes acontinuacion explicados }

\begin{verbatim}

JAVA
import java.io.*;
 
public class c5p6{
    public static void main(String args[]){
 	for(int i = 0; i< 10 ; i++)
 	    {
 		System.out.println(i);
 	    }
 	System.out.println(i);
     }
 }

C++
#include <iostream>
using namespace std;
 
int main(){
   for (int i = 0 ; i<10; i++){
     cout << i;
   }
   cout << i; 
   return 0;
 }

C#
using System;
 
 class c5p6
 {
   static void Main()
   {
     for(int i = 0 ; i < 10 ; i++)
       {
 	Console.WriteLine(i);
       }
     Console.WriteLine(i);
   }
 
 }

\end{verbatim}
En los tres lenguajes, el declarar una variable dentro del for\\
y luego intentar accederla fuera del bloque for, nos genera \\
un error. La variable tiene alcance local dentro del bloque for, y \\
solo puede ser accedida dentro del mismo.

