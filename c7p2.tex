\subsubsection{Pregunta 2: Reescribir el programa del ejercicio 1 en C++,en Java,en C\# y ejecutarlos y comparar los resultados.}
\lstset{language = C++}  %Seteo para poner Codigo de Lenguaje C++
Código traducido de Lenguaje C a lenguaje C++
\begin{lstlisting}[frame = single] %Comienzo del Código

#include <stdio.h>
int fun(int *k);
int main() {
    int i = 10, j = 10, sum1, sum2;
    sum1 = (i / 2) + fun(&i);
    sum2 = fun(&j) + (j / 2);
    printf("sum1 =  %i  \n ",sum1);
    printf("sum2 =  %i ",sum2);
    return 0;
}

int fun(int *k) {
    *k += 4;
    return 3 * (*k) - 1;
}
\end{lstlisting}
OUTPUT:\\
	sum1 =  46\\
	sum2 =  48\\
En C++ a diferencia de C debemos declara el prototipo de la función antes del main, por los resultados obtenidos la regla de asociatividad del lenguaje en C++ es igual a la del Lenguaje C, es decir la Asociatividad es de izquierda a derecha.\\\\
\lstset{language = C}  %Seteo para poner Codigo de Lenguaje C
Código traducido de Lenguaje C++ a lenguaje C\#
\begin{lstlisting}[frame = single] %Comienzo del Código

class Program
    {
        static void Main(string[] args)
        {
            Program p = new Program();
            int i = 10,j=10,sum1,sum2;
            sum1 = (i / 2) + p.fun(ref i);
            sum2 = p.fun(ref j) + (j / 2);
            System.Console.WriteLine(" "+sum1);
            System.Console.WriteLine(" "+sum2);
            Console.Read();
        }
        public int fun(ref int k)
        {
            k = 4 + k;
            return 3*(k) - 1;
        }
    }
\end{lstlisting}
OUTPUT:\\
	sum1 =  46\\
	sum2 =  48\\
El órden de las reglas de asociatividad del lenguaje C\# es el mismo que tiene c++ y c que es el mismo de izquierda a derecha aquí el puntero se reemplaza con la palabra clave "reference" produce un argumento que se pasa por referencia, no por valor.\\\\
\lstset{language = java}  %Seteo para poner Codigo de Lenguaje C#
Código traducido de Lenguaje C a JAVA
\begin{lstlisting}[frame = single] %Comienzo del Código

public class Prueba {
    public static void main(String[] args) {
        Prueba p = new Prueba();
        int i = 10, j = 10, sum1, sum2;
        sum1 = (i/2) + p.fun(i);
        sum2 = p.fun(j) + (j/2);
        System.out.println("Valor 1: "+sum1);
        System.out.println("Valor 2: "+sum2);
    }

    public int fun(int k){
        k += 4;
        return (3 * (k) - 1);
    }
}
\end{lstlisting}
OUTPUT:\\
	sum1 =  46\\
	sum2 =  46\\
Java tiene un tipo de apuntadores o, mejor dicho, referencias. Y no tiene operadores que tenga que poner explícitamente como \*,\^ o \&\&; y no pueden ser manipularlos con operaciones aritméticas o de otro tipo como en C.
Entonces el orden de las reglas de asociatividad del lenguaje en Java.\\