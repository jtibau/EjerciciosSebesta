\documentclass[12pt,oneside]{article}
\usepackage{geometry}                                % See geometry.pdf to learn the layout options. There are lots.
\usepackage{listings}				% Permite utilizar lenguajes de programacion dentro de latex
\geometry{a4paper}                                           % ... or a4paper or a5paper or ... 
%\geometry{landscape}                                % Activate for for rotated page geometry
%\usepackage[parfill]{parskip}                    % Activate to begin paragraphs with an empty line rather than an indent
\usepackage{graphicx}                                % Use pdf, png, jpg, or epsß with pdflatex; use eps in DVI mode
                                                                % TeX will automatically convert eps --> pdf in pdflatex                
\usepackage{amssymb}

\usepackage[spanish]{babel}                        % Permite que partes automáticas del documento aparezcan en castellano.
\usepackage[utf8]{inputenc}                        % Permite escribir tildes y otros caracteres directamente en el .tex
\usepackage[T1]{fontenc}                                % Asegura que el documento resultante use caracteres de una fuente apropiada.

\usepackage{hyperref}                                % Permite poner urls y links dentro del documento

\title{Ejercicios de Programación - Sebesta}
\author{Lenguajes de Programación - ESPOL}

%\date{}                                                        % Activate to display a given date or no date

\begin{document}
\maketitle

\section{Introducción}
Las respuestas propuestas en este repositorio son producto del trabajo de los estudiantes de la materia ``Lenguajes de Programación'' de la ESPOL, correspondientes a las preguntas del libro de Robert Sebesta, Concepts of Programming Languages.

\section{Preguntas y Respuestas}

\subsection{Capítulo 5: Nombres, Enlaces y Alcances.}
\subsubsection{Pregunta 4: Escriba un script en Python con subprogramas triplemente anidados, donde cada subprograma referencie a variables que han sido definidas en otros niveles de la jerarquía}

Escriba su respuesta con claridad. En las secciones de código utilice listings.

\subsection{Pregunta 5: Escriba una función en C que incluya las siguienes líneas de código: \newline
x = 21; \newline
int x; \newline
x = 42; \newline
Corra el programa y explique los resultados. Reescriba el mismo código en C++ y Java.}

Ninguno de los siguientes dos bloques de código compilan en C. En el listing 1, la variable 'x' aún no esta declarada al usarse en la primera asignación.
En el listing 2, hay una nueva declaración de la variable 'x', generando una contradicción.

\lstset{language=C}          % Set your language (you can change the language for each code-block optionally)

\begin{lstlisting}[caption= Pregunta 5 Capítulo 5, label=amb, frame=single]  % Start your code-block
  
#include<stdio.h>

main()
{
	x = 21;
	int x;
	x = 42;

	printf("%d", x);
	getch();

}
\end{lstlisting}
\begin{lstlisting}[caption= Pregunta 5 Capítulo 5, label=amb, frame=single]  % Start your code-block
  
#include<stdio.h>

main()
{
	int x;
	x = 21;
	int x;
	x = 42;

	printf("%d", x);
	getch();

}
\end{lstlisting}

En el listing 3, el programa imprime el valor de 21. Esto se debe a que la variable global (‘x’) es modificada al llamarse a la función ‘foo’. Dentro de esa función se crea, asimismo, una variable ‘x’ que esta solo dentro del scope de la función ‘foo’. Dentro del scope del ‘main’ solo está la variable ‘x’ global y no la ‘x’ de la función ‘foo’, por ende imprime ese valor que se le fue asignado al llamar a ‘foo’.
En el listing 4, hay una nueva declaración de la variable 'x', generando una contradicción.

\lstset{language=C++}          % Set your language (you can change the language for each code-block optionally)

\begin{lstlisting}[caption= Pregunta 5 Capítulo 5, label=amb, frame=single]  % Start your code-block
  
#include <iostream>
using namespace std;

int x ;

void foo () { 
    x =  21 ; 
    int x ; 
    x =  42 ;   
}

int main() {
   foo();
   printf("%d",x);
   return 0;
}
\end{lstlisting}

\begin{lstlisting}[caption= Pregunta 5 Capítulo 5, label=amb, frame=single]  % Start your code-block
  
#include <iostream>
using namespace std;

int main(void) {
	int x;
	x = 21;
	int x;
	x = 42;
	cout << x << endl;
	system("pause");
	return 0;
}
\end{lstlisting}

En el listing 5, lo primero que se realiza es la modificación a la variable global (x) a 21. Luego, se crea otra variable x (dentro de la función) y se le asigna a esta el valor de 42, muy aparte de la variable global, cuyo valor seguirá siendo 21. En el Main, el scope solo incluye a la variable global x, ya que la otra existe solo dentro del scope de la función. Por ende, solo se mostrara la variable global, con el valor que se le asignó al llamar a la función.
En el listing 6, hay una nueva declaración de la variable 'x', generando una contradicción.

\lstset{language=Java}          % Set your language (you can change the language for each code-block optionally)

\begin{lstlisting}[caption= Pregunta 5 Capítulo 5, label=amb, frame=single]  % Start your code-block
  
public class Ejercicio5
{
    static int x;
    public static void main( String[] args )
    {
        Ejercicio5 ej5 = new Ejercicio5();
        ej5.ejercicio5();
        System.out.println(x);
    }
    public void ejercicio5(){
        x = 21;
        int x;
        x = 42;
    }
}
\end{lstlisting}

\begin{lstlisting}[caption= Pregunta 5 Capítulo 5, label=amb, frame=single]  % Start your code-block
  
public static void main(String[] args) {
        int x;
        x = 21;
        int x;
        x = 42;
        
        System.out.println(x);
    }
\end{lstlisting}

\subsubsection{C5 Pregunta6: Sentencia for alcance de la variabe declarada en ella en los tres lenguajes acontinuacion explicados }

\begin{verbatim}

JAVA
import java.io.*;
 
public class c5p6{
    public static void main(String args[]){
 	for(int i = 0; i< 10 ; i++)
 	    {
 		System.out.println(i);
 	    }
 	System.out.println(i);
     }
 }

C++
#include <iostream>
using namespace std;
 
int main(){
   for (int i = 0 ; i<10; i++){
     cout << i;
   }
   cout << i; 
   return 0;
 }

C#
using System;
 
 class c5p6
 {
   static void Main()
   {
     for(int i = 0 ; i < 10 ; i++)
       {
 	Console.WriteLine(i);
       }
     Console.WriteLine(i);
   }
 
 }

\end{verbatim}
En los tres lenguajes, el declarar una variable dentro del for\\
y luego intentar accederla fuera del bloque for, nos genera \\
un error. La variable tiene alcance local dentro del bloque for, y \\
solo puede ser accedida dentro del mismo.


\subsubsection{Pregunta 7: Tres funciones en C donde se declare un arreglo de forma\\
estatica, otra stack y la ultima declaracion como heap}

\begin{verbatim}
STACK 
int main (int argc, char *argv[])
{
 
  int arreglo[3000];
  int i;
  srand(time(NULL));

  for(i=0;i<2999;i++)
  arreglo[i]=1+rand()%100;

  for(i=0;i<2999;i++){
  int doble;
  doble=arreglo[i]*2;
  printf("El doble del numero aleatorio en la posicion %d manejado por pila es: %d\n",i,doble);
 
  }
  return 0;
}

HEAP
int main (int argc, char *argv[])
{
 
  int arreglo[3000]; 
  int i;
  int *p;
  srand(time(NULL));
  


  for(i=0;i<2999;i++){
  p= (int *)malloc(3000*sizeof(int));
  arreglo[i]=1+rand()%100;
  *p=arreglo[i];
  }
 

  for(i=0;i<2999;i++){
  int doble;
  p= (int *)malloc(3000*sizeof(int));
  *p=arreglo[i];
  doble=*p*2;
  printf("El doble del numero aleatorio en la posicion %d manejado por heap es: %d\n",i,doble);
  
  }

  free(p);
  return 0;
}

STATIC

int main (int argc, char *argv[])
{
 
  static int arreglo[3000];
  int i;
  srand(time(NULL));

  for(i=0;i<2999;i++)
  arreglo[i]=1+rand()%100;

  for(i=0;i<2999;i++){
  int doble;
  doble=arreglo[i]*2;
  printf("El doble del numero aleatorio en la posicion %d con arreglo estatico es: %d\n",i,doble);
 
  }
  return 0;
}

\end{verbatim}
Stack tiene un aceso mas rapido, el espacio es manejado por el CPU es limitado y no puede ser redefinido.\\
En el caso del heap nosotros necesitamos manejar la memoria, acceso mas lento y no es limitado\\
Static declaracion unica de una variable que mantiene su dimension a lo largo del tiempo de vida del programa.


\subsection{Capítulo 6}
\subsubsection{C6 Pregunta 1}
\begin{lstlisting}
int main (int argc, char *argv[])
{
	int i,div_int;
	float div_float,j=11.00;
	int arreglo[3];
	arreglo[2]=10;
	div_int=arreglo[2]/2;
	div_float=j/2;
	if(div_int!=div_float){
		printf("%i diferente %.2f",div_int,div_float);
		 Sleep(1000);
	}if(div_int==div_float){
	     printf("%i igual %.2f",div_int,div_float);
		 Sleep(1000);
	}
	
  return 0;
}
 \end{lstlisting}
La compatibilidad de tipos en C es muy variada, si comparamos como en el primer \\
if es claro que apesar de que los dos son numeros cinco los decimales de el segundo\\
valor hacen que la igualdad no se cumpla en caso de haber salido un 5.00 la igualdad \\
se cumplia ignorando que sean de diferentes tipos.Esto hace que cuando comparamos en \\
c el se encargue de algunas conversiones para llevar a caboo comparaciones en el \\
programa, en caso de que cambiemos a int el divfloat nos advertira de la perdida de datos \\
pero no nos generara error alguno.
  
\subsubsection{C6 Pregunta 2}

\begin{lstlisting}
double *multiplicarxdos (double *input) {
  double *twice;
   twice = (double*)malloc(sizeof(double));
  *twice = *input * 2.0;
  return twice;
}

int main (int argc, char *argv[])
{
  int *edad = (int *)malloc(sizeof(int));
  *edad = 23;
  double *salario = (double*)malloc(sizeof(double));
  *salario = 12345.67;
  double *miLista = (double*)malloc(3 * sizeof(double));
  miLista[0] = 1.2;
  miLista[1] = 2.3;
  miLista[2] = 3.4;

  double *twiceSalary = multiplicarxdos(salario);

  printf("El doble del salario es %.3f\n", *twiceSalary);
  Sleep(1000);

  free(edad);
  free(salario);
  free(miLista);
  free(twiceSalary);

  return 0;

}
 \end{lstlisting}
Se hace uso de la funcion free() en C cuando tenemos un acceso a memoria dinamica \\
en el Heap, en este caso como en el ejemplo debemos gestionar la memoria pidiendola \\
y despues liberandola repectivamente.
%\input{c6p7}
\subsection{Capítulo 7}
\subsubsection{Pregunta 1:Ejecute el código dado en el Problema 13(Set de Problemas) en un sistema que soporte C para determinar los valores de sum1 y sum2. Explique los resultados.}
\lstset{language = C}  %Seteo para poner Codigo de Lenguaje C
Código dado en el Problema 13(Set de Problemas)
\begin{lstlisting}[frame = single] %Comienzo del Código

int fun(int *k){
	*k+=4; /* Incrementa el valor de x + 4*/
	return 3 * (*k) -1 /* Retorna */
}

void main(){
	int i=10, j=10,sum1,sum2;
	sum1 =(i/2) + fun(&i);
	sum2 = fun(&j) + (j/2);
}
\end{lstlisting}
OUTPUT:\\
	sum1 = 46\\
	sum2 = 48\\
	Estos resultados se da por las Reglas de Asociatividad del lenguaje C, 
	que es la asociatividad de izquierda a derecha, esto significa que e operador de la izquierda se evaluara primero.\\
	
	Para sum2, la dirección de memoria cambia debido a las operaciones que se realiza
	con la refercencia a la variable fun, por esta razón hay diferentes resultados.\\



\subsubsection{Pregunta 2: Reescribir el programa del ejercicio 1 en C++,en Java,en C\# y ejecutarlos y comparar los resultados.}
\lstset{language = C++}  %Seteo para poner Codigo de Lenguaje C++
Código traducido de Lenguaje C a lenguaje C++
\begin{lstlisting}[frame = single] %Comienzo del Código

#include <stdio.h>
int fun(int *k);
int main() {
    int i = 10, j = 10, sum1, sum2;
    sum1 = (i / 2) + fun(&i);
    sum2 = fun(&j) + (j / 2);
    printf("sum1 =  %i  \n ",sum1);
    printf("sum2 =  %i ",sum2);
    return 0;
}

int fun(int *k) {
    *k += 4;
    return 3 * (*k) - 1;
}
\end{lstlisting}
OUTPUT:\\
	sum1 =  46\\
	sum2 =  48\\
En C++ a diferencia de C debemos declara el prototipo de la función antes del main, por los resultados obtenidos la regla de asociatividad del lenguaje en C++ es igual a la del Lenguaje C, es decir la Asociatividad es de izquierda a derecha.\\\\
\lstset{language = C}  %Seteo para poner Codigo de Lenguaje C
Código traducido de Lenguaje C++ a lenguaje C\#
\begin{lstlisting}[frame = single] %Comienzo del Código

class Program
    {
        static void Main(string[] args)
        {
            Program p = new Program();
            int i = 10,j=10,sum1,sum2;
            sum1 = (i / 2) + p.fun(ref i);
            sum2 = p.fun(ref j) + (j / 2);
            System.Console.WriteLine(" "+sum1);
            System.Console.WriteLine(" "+sum2);
            Console.Read();
        }
        public int fun(ref int k)
        {
            k = 4 + k;
            return 3*(k) - 1;
        }
    }
\end{lstlisting}
OUTPUT:\\
	sum1 =  46\\
	sum2 =  48\\
El órden de las reglas de asociatividad del lenguaje C\# es el mismo que tiene c++ y c que es el mismo de izquierda a derecha aquí el puntero se reemplaza con la palabra clave "reference" produce un argumento que se pasa por referencia, no por valor.\\\\
\lstset{language = java}  %Seteo para poner Codigo de Lenguaje C#
Código traducido de Lenguaje C a JAVA
\begin{lstlisting}[frame = single] %Comienzo del Código

public class Prueba {
    public static void main(String[] args) {
        Prueba p = new Prueba();
        int i = 10, j = 10, sum1, sum2;
        sum1 = (i/2) + p.fun(i);
        sum2 = p.fun(j) + (j/2);
        System.out.println("Valor 1: "+sum1);
        System.out.println("Valor 2: "+sum2);
    }

    public int fun(int k){
        k += 4;
        return (3 * (k) - 1);
    }
}
\end{lstlisting}
OUTPUT:\\
	sum1 =  46\\
	sum2 =  46\\
Java tiene un tipo de apuntadores o, mejor dicho, referencias. Y no tiene operadores que tenga que poner explícitamente como \*,\^ o \&\&; y no pueden ser manipularlos con operaciones aritméticas o de otro tipo como en C.
Entonces el orden de las reglas de asociatividad del lenguaje en Java.\\
\subsubsection{C7 Pregunta 1}

\begin{lstlisting}

def pagoSemestral():

       pagoxmateria=120
       materiasxtomar=int(input("Cuantas Materias va a tomar este Semestre:   \n"))
       promediocarrera=int(input("Cual es su promedio general:    \n"))
       
       if(9<=promediocarrera):
          valorxpagar=(pagoxmateria*materiasxtomar)/2
          print("Tu valor a pagar en es: $"+str(valorxpagar))
       elif(8<=promediocarrera<9):
          valorxpagar=(pagoxmateria*materiasxtomar)-(pagoxmateria*materiasxtomar)*(20/100)
          print("Tu valor a pagar es: $"+str(valorxpagar))
       else:
          valorxpagar=(pagoxmateria*materiasxtomar)
          print("Tu valor a pagar es: $"+str(valorxpagar))
             
pagoSemestral()

 \end{lstlisting}
En Python como en la mayoria de lenguajes existe la regla de precedencia izquierda a derecha \\
con excepcion de la exponenciacion que es lo contrario. En este ejemplo se calcula el valor a pagar \\
en dolares para registrarse en un semestre de una universidad X.

\subsubsection{C7 Pregunta 4}
\begin{lstlisting}
 public class JavaApplication2 {
     final static int num=5;
     static int a=5;
    
     public static void main(String[] args) {          
           a = fun1()+a; 
           System.out.println(a); 
     }
 
     static int fun1() {
             a = 17;
             return 3;
        }
 
 }
 
 \end{lstlisting}
 OUTPUT\\
 20\\
 Como podemos ver en el main en la 1era linea, fun1() retorna 3 y actualiza la variable global 'a'=17 y luego suma 17+3 asignando 20 a 'a'. Esto se debe a que en JAVA el operador '+' tiene asociatividad desde la izquierda.\\
 En cambio si fuese:  a = a + fun1();. En este caso primero 'a'=5 + fun1() que retorne 3; dando una asignación de 8 a la variable a.
\subsubsection{C7 Pregunta 5}
\begin{lstlisting}

 int fun1();
 
 extern int a = 10;
 void main(){
 	
 	a = fun1()+a;  // a = a+fun1();
 	printf("\%d", a);
 	getch();
 }
 
 int fun1() {
 	a = 17;
 	return 3;
 }
 \end{lstlisting}
Como podemos ver en el main en la 1era linea, comparado con JAVA Y Si Shard en vez de que fun1() retorne 3 y actualize la variable global 'a' a 17,y luego sumarlos y asignarle 20 a 'a'. En C++, sea a = fun1()+a  ó  a = a+fun1(), llamado de función en la izq o derecha del operador en este caso '+'; siempre al llamar fun1(), este va actualizar la variable global a =17 y luego va sumarle 3, asignando un valor de 20 a  la variable 'a'. 
\subsubsection{C7 Pregunta 6}
\begin{lstlisting}
 class Program
     {
        static int a = 5;
         static void Main(string[] args)
         {
             a = a +fun1();
             Console.WriteLine(a);
             Console.ReadLine();
        }
 
        static int fun1()
         {
             a = 17;
             return 3;
         }
     }
 \end{lstlisting}
 OUTPUT\\
 8\\ 
Como podemos ver en Si Sharp también se cumple la regla de la asociativad. En este caso el operador '+' asoicativdad desde la izquierda. Primero a=5, luego le suma 3; cuyo resultado que es 8 se le asigna a la variable a.\\
En cambio si fuese:  a = fun1()+a. En ese caso fun1() retorna 3 y actualiza la variable global 'a'=17 y luego suma 17+3 asignando 20 a la varaible a.
  
%\input{c7p9}
\subsection{Capítulo 8}
\subsubsection{C8 Pregunta 3}

\lstset{language=JAva} 
 \begin{lstlisting}

 
 public class chapterEightExThree {

     
     public static void main(String args[]){
         int j=0; int k=0;
         
         switch(k){
             case 1: j= 2 * k -1;
             case 2: j = 2 * k - 1;
             case 3: j = 3 * k + 1;
             case 4: j = 4 * k - 1;
             case 5: j = 3 * k + 1;
             case 6: j = k - 2;
             case 7: j = k - 2;
             case 8: j = k - 2;
             default: System.out.println("Fuera de rango");
                 
         
         }
         
     
     }
 }
 \end{lstlisting}
\subsubsection{C8 Pregunta 4}

\lstset{language=C}      
 \begin{lstlisting}
 #include <stdio.h>
 #include <string.h>
 
 main()
 {
 int i=0;
 int j=-3;
 int key=j+2;
 
 for(i=0;i<10;i++){
     if((key==3)||(key ==2)){
 		j--;
 	}else{
 		if(key==0){
 			j=j+2;
 		
 		}else{
 			j=0;
 			if(j>0){
 				break;
 			}else{
 				j=3-i;
 			}
 		}
 	}		
 }
 }
 
 \end{lstlisting}
\subsubsection{C8 Pregunta 5}

\lstset{language=JAva}       
 \begin{lstlisting}

 public class chapterEightExFive {
     
     public static void main(String args[]){
         boolean encontrado =false;
         int n=5; //suponemos la dimesion 5
         int[][] x = new int[n][n];
         
         for(int i=0;i<=n;i++){
             int acum=0;
             for(int j=0;j<=n;j++){
                 if(x[i][j]==0){
                     acum++;
                 }
                 if(acum==n && !encontrado){
                     System.out.println("Encontrad");
                     encontrado=true;
                 }
             }      
         }
         
     }
 }
 }
 \end{lstlisting}
 Aunque el codigo presentado en el programa es mas corto, con seguridad el mismo ejercicio transcrito a lenguaje java presenta mayor legibilidad
\subsection{Capítulo 9}
%\input{c9p1}
\subsubsection{C9 Pregunta 5}

 \lstset{language=JAva}    
 \begin{lstlisting}

 public class chapterNineExFive {
 	static int m =100;
 	public static int[][] static_matriz1 = new int[m][m];
 	public static int[][] static_matriz2 = new int[m][m];
 	public static int[][] static_resul = new int[m][m];
 	
 	
 	public chapterNineExFive(){
 
 	}
 	
 	public void matricesRandom(int[][] a, int[][] b){
 		for(int fil=0;fil<m;fil++){
 			for(int col=0;col<m;col++){
 				a[fil][col] = (int)(Math.random()*m)+1;
 				b[fil][col] = (int)(Math.random()*m)+1;
 				
 			}
 		}
 	}
 	
 	public void operacionesStatic(){
 		this.matricesRandom(static_matriz1, static_matriz2);
 		
 		 for(int i = 0; i < m; i++){
 	            for (int j = 0; j < m; j++){
 	                for (int k = 0; k < m; k++){
 	                    static_resul[i][j] += 
                             static_matriz1[i][k] * static_matriz2[k][j];
 	                }
 	            }
 	        }

 	}
 	
 	public void operacionesDinamic(){
 		int[][] dinamic_matriz1 = new int[m][m];
 		int[][] dinamic_matriz2 = new int[m][m];
 		int[][] dinamic_resul = new int[m][m];
 		
 		this.matricesRandom(dinamic_matriz1, dinamic_matriz2);
 		
 		 for(int i = 0; i < m; i++){
 	            for (int j = 0; j < m; j++){
 	                for (int k = 0; k < m; k++){
 	                	dinamic_resul[i][j] += 
                                  dinamic_matriz1[i][k] * dinamic_matriz2[k][j];
 	                }
 	            }
 	        }
 	
 	}
 
 
 	public static void main(String args[]){
             
 		long startTime = System.nanoTime();
 		new  chapterNineExFive().operacionesStatic();
 		long endTime = System.nanoTime();
 		long duration = endTime - startTime;
 		System.out.println("Tiempo Estatico: " + duration);
                
                 
                 long startTimeD = System.nanoTime();
 		new  chapterNineExFive().operacionesDinamic();
 		long endTimeD = System.nanoTime();
 		long durationD = endTimeD - startTimeD;
                 System.out.println("Tiempo Dinamico: " + durationD);
 	}
 }
 \end{lstlisting}
 El resultado de este operacion fue una media alrededor de 14,4u milis para la operacion statica y alrededor de  13,6u para la operacion dinamica. Aunque la diferencia es casi minima, y casi imperceptible el metodo dinamico resulto ser mas rapido que el estatico.

% Continuar con los siguientes capítulos y ejercicios:
% Ch6: 1, 2, 7
% Ch7: 1 - 6, 9
% Ch8: 3, 4, 5
% Ch9: 1, 5
% Recuerden que todos corresponden a las secciones de "Programming Exercises".

\end{document}
