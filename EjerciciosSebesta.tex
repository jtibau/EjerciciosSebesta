\documentclass[12pt,oneside]{article}
\usepackage{geometry}                                % See geometry.pdf to learn the layout options. There are lots.
\usepackage{listings}				% Permite utilizar lenguajes de programacion dentro de latex
\geometry{a4paper}                                           % ... or a4paper or a5paper or ... 
%\geometry{landscape}                                % Activate for for rotated page geometry
%\usepackage[parfill]{parskip}                    % Activate to begin paragraphs with an empty line rather than an indent
\usepackage{graphicx}                                % Use pdf, png, jpg, or epsß with pdflatex; use eps in DVI mode
                                                                % TeX will automatically convert eps --> pdf in pdflatex                
\usepackage{amssymb}

\usepackage[spanish]{babel}                        % Permite que partes automáticas del documento aparezcan en castellano.
\usepackage[utf8]{inputenc}                        % Permite escribir tildes y otros caracteres directamente en el .tex
\usepackage[T1]{fontenc}                                % Asegura que el documento resultante use caracteres de una fuente apropiada.

\usepackage{hyperref}                                % Permite poner urls y links dentro del documento

\title{Ejercicios de Programación - Sebesta}
\author{Lenguajes de Programación - ESPOL}
%\date{}                                                        % Activate to display a given date or no date

\begin{document}
\maketitle

\section{Introducción}
Las respuestas propuestas en este repositorio son producto del trabajo de los estudiantes de la materia ``Lenguajes de Programación'' de la ESPOL, correspondientes a las preguntas del libro de Robert Sebesta, Concepts of Programming Languages.

\section{Preguntas y Respuestas}

\subsection{Capítulo 5: Nombres, Enlaces y Alcances.}
\input{c5p4}
%\subsubsection{Pregunta 4: Escriba un script en Python con subprogramas triplemente anidados, donde cada subprograma referencie a variables que han sido definidas en otros niveles de la jerarquía}

Escriba su respuesta con claridad. En las secciones de código utilice listings.

Capitulo 5 pregunta 5
En caso de querer llamar en nuestro programa números pseudoaleatorios, que han sido generados por un subprograma se necesita del history-sensitivy para almacenar la última iteración de la variable localmente. 
Ejemplo de un subprograma generando numeros pseudoaleatorios:

 public void generarSerieDeAleatorios () \{  \\

        Random numAleatorio;

        numAleatorio = new Random ();

        for (int i=0; i < serieAleatoria.size(); i++)  \{

        serieAleatoria.set(i, numAleatorio.nextInt(1000) );

         \}
\subsection{Capítulo 7: Expresiones y Asignaciones de sentencias.}
\subsubsection{Pregunta 1:Ejecute el código dado en el Problema 13(Set de Problemas) en un sistema que soporte C para determinar los valores de sum1 y sum2. Explique los resultados.}
\lstset{language = C}  %Seteo para poner Codigo de Lenguaje C
Código dado en el Problema 13(Set de Problemas)
\begin{lstlisting}[frame = single] %Comienzo del Código

int fun(int *k){
	*k+=4; /* Incrementa el valor de x + 4*/
	return 3 * (*k) -1 /* Retorna */
}

void main(){
	int i=10, j=10,sum1,sum2;
	sum1 =(i/2) + fun(&i);
	sum2 = fun(&j) + (j/2);
}
\end{lstlisting}
OUTPUT:\\
	sum1 = 46\\
	sum2 = 48\\
	Estos resultados se da por las Reglas de Asociatividad del lenguaje C, 
	que es la asociatividad de izquierda a derecha, esto significa que e operador de la izquierda se evaluara primero.\\
	
	Para sum2, la dirección de memoria cambia debido a las operaciones que se realiza
	con la refercencia a la variable fun, por esta razón hay diferentes resultados.\\




%\subsubsection{Pregunta 6: Escriba programas de prueba en C++, Java, y C\# para determinar el alcance de una variable declarada en una sentencia for. Específicamente, el código debe determinar si esta variable es visible después del cuerpo de la sentencia for.}

\lstinputlisting[language=C++, frame=single,caption=Lenguaje C++]{c5p6.cpp}

\lstinputlisting[language=Java,frame=single, caption=Lenguaje Java]{c5p6.java}

\lstinputlisting[language=C,frame=single, caption= Lenguaje C \#]{c5p6.cs}

Podemos observar que en todos los tres lenguajes, el declarar una variable dentro del for y luego intentar accederla fuera del bloque for, nos genera un error. Esto se debe a que la variable tiene alcance local dentro del bloque for, y solo puede ser accedida dentro del mismo.

%\input{c5p7}

% Continuar con los siguientes capítulos y ejercicios:
% Ch6: 1, 2, 7
% Ch7: 1 - 6, 9
% Ch8: 3, 4, 5
% Ch9: 1, 5
% Recuerden que todos corresponden a las secciones de "Programming Exercises".

\end{document}
