\documentclass[12pt,oneside]{article}
\usepackage{geometry}                                % See geometry.pdf to learn the layout options. There are lots.
\usepackage{listings}				% Perm ite utilizar lenguajes de programacion dentro de latex
\geometry{a4paper}                                           % ... or a4paper or a5paper or ... 
%\geometry{landscape}                                % Activate for for rotated page geometry
%\usepackage[parfill]{parskip}                    % Activate to begin paragraphs with an empty line rather than an indent
\usepackage{graphicx}                                % Use pdf, png, jpg, or epsß with pdflatex; use eps in DVI mode
                                                                % TeX will automatically convert eps --> pdf in pdflatex                
\usepackage{amssymb}

\usepackage[spanish]{babel}                        % Permite que partes automáticas del documento aparezcan en castellano.
\usepackage[utf8]{inputenc}                        % Permite escribir tildes y otros caracteres directamente en el .tex
\usepackage[T1]{fontenc}                                % Asegura que el documento resultante use caracteres de una fuente apropiada.

\usepackage{hyperref}                                % Permite poner urls y links dentro del documento
\usepackage{listings}
\title{Ejercicios de Programación - Sebesta}
\author{Lenguajes de Programación - ESPOL}

%\date{}                                                        % Activate to display a given date or no date

\begin{document}
\maketitle

\section{Introducción}
Las respuestas propuestas en este repositorio son producto del trabajo de los estudiantes de la materia ``Lenguajes de Programación'' de la ESPOL, correspondientes a las preguntas del libro de Robert Sebesta, Concepts of Programming Languages.

\section{Preguntas y Respuestas}

\subsection{Capítulo 5: Nombres, Enlaces y Alcances.}
\subsubsection{Pregunta 4: Escriba un script en Python con subprogramas triplemente anidados, donde cada subprograma referencie a variables que han sido definidas en otros niveles de la jerarquía}

Escriba su respuesta con claridad. En las secciones de código utilice listings.

\subsection{Pregunta 5: Escriba una función en C que incluya las siguienes líneas de código: \newline
x = 21; \newline
int x; \newline
x = 42; \newline
Corra el programa y explique los resultados. Reescriba el mismo código en C++ y Java.}

Ninguno de los siguientes dos bloques de código compilan en C. En el listing 1, la variable 'x' aún no esta declarada al usarse en la primera asignación.
En el listing 2, hay una nueva declaración de la variable 'x', generando una contradicción.

\lstset{language=C}          % Set your language (you can change the language for each code-block optionally)

\begin{lstlisting}[caption= Pregunta 5 Capítulo 5, label=amb, frame=single]  % Start your code-block
  
#include<stdio.h>

main()
{
	x = 21;
	int x;
	x = 42;

	printf("%d", x);
	getch();

}
\end{lstlisting}
\begin{lstlisting}[caption= Pregunta 5 Capítulo 5, label=amb, frame=single]  % Start your code-block
  
#include<stdio.h>

main()
{
	int x;
	x = 21;
	int x;
	x = 42;

	printf("%d", x);
	getch();

}
\end{lstlisting}

En el listing 3, el programa imprime el valor de 21. Esto se debe a que la variable global (‘x’) es modificada al llamarse a la función ‘foo’. Dentro de esa función se crea, asimismo, una variable ‘x’ que esta solo dentro del scope de la función ‘foo’. Dentro del scope del ‘main’ solo está la variable ‘x’ global y no la ‘x’ de la función ‘foo’, por ende imprime ese valor que se le fue asignado al llamar a ‘foo’.
En el listing 4, hay una nueva declaración de la variable 'x', generando una contradicción.

\lstset{language=C++}          % Set your language (you can change the language for each code-block optionally)

\begin{lstlisting}[caption= Pregunta 5 Capítulo 5, label=amb, frame=single]  % Start your code-block
  
#include <iostream>
using namespace std;

int x ;

void foo () { 
    x =  21 ; 
    int x ; 
    x =  42 ;   
}

int main() {
   foo();
   printf("%d",x);
   return 0;
}
\end{lstlisting}

\begin{lstlisting}[caption= Pregunta 5 Capítulo 5, label=amb, frame=single]  % Start your code-block
  
#include <iostream>
using namespace std;

int main(void) {
	int x;
	x = 21;
	int x;
	x = 42;
	cout << x << endl;
	system("pause");
	return 0;
}
\end{lstlisting}

En el listing 5, lo primero que se realiza es la modificación a la variable global (x) a 21. Luego, se crea otra variable x (dentro de la función) y se le asigna a esta el valor de 42, muy aparte de la variable global, cuyo valor seguirá siendo 21. En el Main, el scope solo incluye a la variable global x, ya que la otra existe solo dentro del scope de la función. Por ende, solo se mostrara la variable global, con el valor que se le asignó al llamar a la función.
En el listing 6, hay una nueva declaración de la variable 'x', generando una contradicción.

\lstset{language=Java}          % Set your language (you can change the language for each code-block optionally)

\begin{lstlisting}[caption= Pregunta 5 Capítulo 5, label=amb, frame=single]  % Start your code-block
  
public class Ejercicio5
{
    static int x;
    public static void main( String[] args )
    {
        Ejercicio5 ej5 = new Ejercicio5();
        ej5.ejercicio5();
        System.out.println(x);
    }
    public void ejercicio5(){
        x = 21;
        int x;
        x = 42;
    }
}
\end{lstlisting}

\begin{lstlisting}[caption= Pregunta 5 Capítulo 5, label=amb, frame=single]  % Start your code-block
  
public static void main(String[] args) {
        int x;
        x = 21;
        int x;
        x = 42;
        
        System.out.println(x);
    }
\end{lstlisting}

\subsubsection{C5 Pregunta6: Sentencia for alcance de la variabe declarada en ella en los tres lenguajes acontinuacion explicados }

\begin{verbatim}

JAVA
import java.io.*;
 
public class c5p6{
    public static void main(String args[]){
 	for(int i = 0; i< 10 ; i++)
 	    {
 		System.out.println(i);
 	    }
 	System.out.println(i);
     }
 }

C++
#include <iostream>
using namespace std;
 
int main(){
   for (int i = 0 ; i<10; i++){
     cout << i;
   }
   cout << i; 
   return 0;
 }

C#
using System;
 
 class c5p6
 {
   static void Main()
   {
     for(int i = 0 ; i < 10 ; i++)
       {
 	Console.WriteLine(i);
       }
     Console.WriteLine(i);
   }
 
 }

\end{verbatim}
En los tres lenguajes, el declarar una variable dentro del for\\
y luego intentar accederla fuera del bloque for, nos genera \\
un error. La variable tiene alcance local dentro del bloque for, y \\
solo puede ser accedida dentro del mismo.


\subsubsection{Pregunta 7: Tres funciones en C donde se declare un arreglo de forma\\
estatica, otra stack y la ultima declaracion como heap}

\begin{verbatim}
STACK 
int main (int argc, char *argv[])
{
 
  int arreglo[3000];
  int i;
  srand(time(NULL));

  for(i=0;i<2999;i++)
  arreglo[i]=1+rand()%100;

  for(i=0;i<2999;i++){
  int doble;
  doble=arreglo[i]*2;
  printf("El doble del numero aleatorio en la posicion %d manejado por pila es: %d\n",i,doble);
 
  }
  return 0;
}

HEAP
int main (int argc, char *argv[])
{
 
  int arreglo[3000]; 
  int i;
  int *p;
  srand(time(NULL));
  


  for(i=0;i<2999;i++){
  p= (int *)malloc(3000*sizeof(int));
  arreglo[i]=1+rand()%100;
  *p=arreglo[i];
  }
 

  for(i=0;i<2999;i++){
  int doble;
  p= (int *)malloc(3000*sizeof(int));
  *p=arreglo[i];
  doble=*p*2;
  printf("El doble del numero aleatorio en la posicion %d manejado por heap es: %d\n",i,doble);
  
  }

  free(p);
  return 0;
}

STATIC

int main (int argc, char *argv[])
{
 
  static int arreglo[3000];
  int i;
  srand(time(NULL));

  for(i=0;i<2999;i++)
  arreglo[i]=1+rand()%100;

  for(i=0;i<2999;i++){
  int doble;
  doble=arreglo[i]*2;
  printf("El doble del numero aleatorio en la posicion %d con arreglo estatico es: %d\n",i,doble);
 
  }
  return 0;
}

\end{verbatim}
Stack tiene un aceso mas rapido, el espacio es manejado por el CPU es limitado y no puede ser redefinido.\\
En el caso del heap nosotros necesitamos manejar la memoria, acceso mas lento y no es limitado\\
Static declaracion unica de una variable que mantiene su dimension a lo largo del tiempo de vida del programa.


\subsection{Capítulo 7: Expressions and Assignment Statements}
\subsubsection{Pregunta 1:Ejecute el código dado en el Problema 13(Set de Problemas) en un sistema que soporte C para determinar los valores de sum1 y sum2. Explique los resultados.}
\lstset{language = C}  %Seteo para poner Codigo de Lenguaje C
Código dado en el Problema 13(Set de Problemas)
\begin{lstlisting}[frame = single] %Comienzo del Código

int fun(int *k){
	*k+=4; /* Incrementa el valor de x + 4*/
	return 3 * (*k) -1 /* Retorna */
}

void main(){
	int i=10, j=10,sum1,sum2;
	sum1 =(i/2) + fun(&i);
	sum2 = fun(&j) + (j/2);
}
\end{lstlisting}
OUTPUT:\\
	sum1 = 46\\
	sum2 = 48\\
	Estos resultados se da por las Reglas de Asociatividad del lenguaje C, 
	que es la asociatividad de izquierda a derecha, esto significa que e operador de la izquierda se evaluara primero.\\
	
	Para sum2, la dirección de memoria cambia debido a las operaciones que se realiza
	con la refercencia a la variable fun, por esta razón hay diferentes resultados.\\



\subsubsection{Pregunta 2: Reescribir el programa del ejercicio 1 en C++,en Java,en C\# y ejecutarlos y comparar los resultados.}
\lstset{language = C++}  %Seteo para poner Codigo de Lenguaje C++
Código traducido de Lenguaje C a lenguaje C++
\begin{lstlisting}[frame = single] %Comienzo del Código

#include <stdio.h>
int fun(int *k);
int main() {
    int i = 10, j = 10, sum1, sum2;
    sum1 = (i / 2) + fun(&i);
    sum2 = fun(&j) + (j / 2);
    printf("sum1 =  %i  \n ",sum1);
    printf("sum2 =  %i ",sum2);
    return 0;
}

int fun(int *k) {
    *k += 4;
    return 3 * (*k) - 1;
}
\end{lstlisting}
OUTPUT:\\
	sum1 =  46\\
	sum2 =  48\\
En C++ a diferencia de C debemos declara el prototipo de la función antes del main, por los resultados obtenidos la regla de asociatividad del lenguaje en C++ es igual a la del Lenguaje C, es decir la Asociatividad es de izquierda a derecha.\\\\
\lstset{language = C}  %Seteo para poner Codigo de Lenguaje C
Código traducido de Lenguaje C++ a lenguaje C\#
\begin{lstlisting}[frame = single] %Comienzo del Código

class Program
    {
        static void Main(string[] args)
        {
            Program p = new Program();
            int i = 10,j=10,sum1,sum2;
            sum1 = (i / 2) + p.fun(ref i);
            sum2 = p.fun(ref j) + (j / 2);
            System.Console.WriteLine(" "+sum1);
            System.Console.WriteLine(" "+sum2);
            Console.Read();
        }
        public int fun(ref int k)
        {
            k = 4 + k;
            return 3*(k) - 1;
        }
    }
\end{lstlisting}
OUTPUT:\\
	sum1 =  46\\
	sum2 =  48\\
El órden de las reglas de asociatividad del lenguaje C\# es el mismo que tiene c++ y c que es el mismo de izquierda a derecha aquí el puntero se reemplaza con la palabra clave "reference" produce un argumento que se pasa por referencia, no por valor.\\\\
\lstset{language = java}  %Seteo para poner Codigo de Lenguaje C#
Código traducido de Lenguaje C a JAVA
\begin{lstlisting}[frame = single] %Comienzo del Código

public class Prueba {
    public static void main(String[] args) {
        Prueba p = new Prueba();
        int i = 10, j = 10, sum1, sum2;
        sum1 = (i/2) + p.fun(i);
        sum2 = p.fun(j) + (j/2);
        System.out.println("Valor 1: "+sum1);
        System.out.println("Valor 2: "+sum2);
    }

    public int fun(int k){
        k += 4;
        return (3 * (k) - 1);
    }
}
\end{lstlisting}
OUTPUT:\\
	sum1 =  46\\
	sum2 =  46\\
Java tiene un tipo de apuntadores o, mejor dicho, referencias. Y no tiene operadores que tenga que poner explícitamente como \*,\^ o \&\&; y no pueden ser manipularlos con operaciones aritméticas o de otro tipo como en C.
Entonces el orden de las reglas de asociatividad del lenguaje en Java.\\
\begin{itemize}
\item {\bf Pregunta 3} \\\\
Escribe un programa en tu lenguaje favorito que determine y muestre la precedencia y asociatividad de sus operadores aritmeticos y booleanos.\\
Lenguaje Java\\
\begin{lstlisting}[frame=single]  % Start your code-block
using System;
public static void main(String[] args) \{
        
        float a,b,c,d,e,res1,res2, res3;
        res1=res2=res3=0;
        
        a=8;
        b=4;
        c=3;
        d=5;
        e=0;
        res1=a/b*c+d;
        res2=a*b/c+d;
        if(e!=0)\{
            res3=e*b*(c+d);
        \}
        System.out.println("El resultado 1 es: "+res1);
        System.out.println("El resultado 2 es: "+res2);
        System.out.println("El resultado 3 es: "+res3);
\}

\end{lstlisting}

Se sabe que en Java: La Exponenciación tiene precedencia 1(mayor), cambio de signo(-) e identidad(+) tienen precedencia 2, division y multiplicacion tienen precedencia 3, la suma y resta tienen precedencia 4 ,etc.\\
Los operadores también pueden tener la misma precedencia. En este caso, la asociatividad determina el orden en que deben actuar los operadores. \\
La asociatividad puede ser de izquierda a derecha o de derecha a izquierda.\\
En el resultado 1(res1), primero se resuelve los operadores de mayor precedencia. En este caso, tenemos la division y multiplicacion de igual precedencia.\\
Como ambas tienen la misma precedencia y asociatividad desde la izquierda, lo que significa que los operadores de la izquierda se procesan (division)antes que los operadores de la derecha(multiplicacion), quiere decir:\\
Primero, realiza a/b, luego ese resultado * c, y finalmente el resultado de la multiplición se le suma d. \\\\
En el resultado 2 (res2), como (/,*) tienen misma precedencia, primero se resuelve el operador de la izquierda (*), luego la división. Finalmente la suma de menor precedencia\\\\
En el resultado 3, se hace una evaluación de corto circuito. Cada vez que e sea diferente de 0, realiza la operación; caso contrario es 0\\\\\\

OUTPUT\\
El resultado 1 es: 11.0\\
El resultado 2 es: 15.666667\\
El resultado 3 es: 0.0\\

\item {\bf Pregunta 4} \\\\
Escribir un programa en Java que exponga la regla para el orden de evaluación de operando cuando uno de los operandos es un método call.\\
Lenguaje Java\\
\begin{lstlisting}[frame=single]  % Start your code-block
public class JavaApplication2 \{
    final static int num=5;
    static int a=5;
    /**
     * @param args the command line arguments
     */
    public static void main(String[] args) \{
        
          a = fun1()+a; 
          System.out.println(a); 
    \}

    static int fun1() \{
            a = 17;
            return 3;
       \}

\}

\end{lstlisting}
OUTPUT\\
20\\
Como podemos ver en el main en la 1era linea, fun1() retorna 3 y actualiza la variable global 'a'=17 y luego suma 17+3 asignando 20 a 'a'. Esto se debe a que en JAVA el operador '+' tiene asociatividad desde la izquierda.\\
En cambio si fuese:  a = a + fun1();. En este caso primero 'a'=5 + fun1() que retorne 3; dando una asignación de 8 a la variable a.

\item {\bf Pregunta 5} \\\\
Lenguaje C++
\begin{lstlisting}[frame=single]  % Start your code-block

#include<stdio.h>
#include<conio.h>
#include<stdlib.h>
#include<windows.h>
#include<math.h>

int fun1();

extern int a = 10;
void main()\{
	
	a = fun1()+a;  // a = a+fun1();
	printf("\%d", a);
	getch();
\}

int fun1() \{
	a = 17;
	return 3;
\}
\end{lstlisting}

OUTPUT\\
20\\
Como podemos ver en el main en la 1era linea, comparado con JAVA Y Si Shard en vez de que fun1() retorne 3 y actualize la variable global 'a' a 17,y luego sumarlos y asignarle 20 a 'a'. En C++, sea a = fun1()+a  ó  a = a+fun1(), llamado de función en la izq o derecha del operador en este caso '+'; siempre al llamar fun1(), este va actualizar la variable global a =17 y luego va sumarle 3, asignando un valor de 20 a  la variable 'a'. 


\item {\bf Pregunta 6} \\\\
Lenguaje Si Sharp\\
\begin{lstlisting}[frame=single]  % Start your code-block
class Program
    \{
        static int a = 5;
        static void Main(string[] args)
        \{
            a = a +fun1();
            Console.WriteLine(a);
            Console.ReadLine();
       \}

        static int fun1()
        \{
            a = 17;
            return 3;
        \}
    }
\end{lstlisting}
OUTPUT\\
8\\ 
Como podemos ver en Si Sharp también se cumple la regla de la asociativad. En este caso el operador '+' asoicativdad desde la izquierda. Primero a=5, luego le suma 3; cuyo resultado que es 8 se le asigna a la variable a.\\
En cambio si fuese:  a = fun1()+a. En ese caso fun1() retorna 3 y actualiza la variable global 'a'=17 y luego suma 17+3 asignando 20 a la varaible a.


\end{itemize}

\subsection{Capítulo 9: SubProgramas.}
\subsubsection{Pregunta 4:Escribir en C\# un programa que pase por referencia un literal a un subprograma, que intenta cambiar el parametro. Dado la filosofia total de C\#, explique los resultados.}
\lstset{language = C}  %Seteo para poner Codigo de Lenguaje C
Código creado para la explicación en C\#.
\begin{lstlisting}[frame = single] %Comienzo del Código

 static void Main(string[] args)
        {
            String a;
            int b= -1;
            Program p = new Program();
            System.Console.Write("Ingrese numero:\n");
            a = Console.ReadLine();
            System.Console.Write("El numero es: " +a);
            System.Console.Write("\n");
            b = Convert.ToInt32(a);
            p.fun(ref b);
            System.Console.Write("El numero enviado a la funcion por referencia es: "+b);
            System.Console.Read();
            //System.Console.ReadKey();
        }

        public void fun (ref int x){
            x = x + 5;  
        }
\end{lstlisting}
OUTPUT:\\
	Ingrese numero: 6\\
	El numero es: 6\\
	El numero enviado a la funcion por referencia es:11\\
	El lenguague C\# recibe por defecto cuando se ingresa por teclado un Strign a este valor hay que convertirlo a un entero para enviarlo a nuestra funcion con la que queremos mostrar lo que deseamos.\\
 Lo que queremos mostrar es el cambio que se produce al valor de una variable siendo enviada a un subrprograma por referencia.\\


%\subsubsection{C5 Pregunta6: Sentencia for alcance de la variabe declarada en ella en los tres lenguajes acontinuacion explicados }

\begin{verbatim}

JAVA
import java.io.*;
 
public class c5p6{
    public static void main(String args[]){
 	for(int i = 0; i< 10 ; i++)
 	    {
 		System.out.println(i);
 	    }
 	System.out.println(i);
     }
 }

C++
#include <iostream>
using namespace std;
 
int main(){
   for (int i = 0 ; i<10; i++){
     cout << i;
   }
   cout << i; 
   return 0;
 }

C#
using System;
 
 class c5p6
 {
   static void Main()
   {
     for(int i = 0 ; i < 10 ; i++)
       {
 	Console.WriteLine(i);
       }
     Console.WriteLine(i);
   }
 
 }

\end{verbatim}
En los tres lenguajes, el declarar una variable dentro del for\\
y luego intentar accederla fuera del bloque for, nos genera \\
un error. La variable tiene alcance local dentro del bloque for, y \\
solo puede ser accedida dentro del mismo.



        




% Continuar con los siguientes capítulos y ejercicios:
% Ch6: 1, 2, 7
% Ch7: 1 - 6, 9
% Ch8: 3, 4, 5
% Ch9: 1, 5
% Recuerden que todos corresponden a las secciones de "Programming Exercises".

\end{document}
