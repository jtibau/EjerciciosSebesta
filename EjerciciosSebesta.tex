\documentclass[12pt,oneside]{article}
\usepackage{geometry}                                % See geometry.pdf to learn the layout options. There are lots.
\usepackage{listings}				% Permite utilizar lenguajes de programacion dentro de latex
\geometry{a4paper}                                           % ... or a4paper or a5paper or ... 
%\geometry{landscape}                                % Activate for for rotated page geometry
%\usepackage[parfill]{parskip}                    % Activate to begin paragraphs with an empty line rather than an indent
\usepackage{graphicx}                                % Use pdf, png, jpg, or epsß with pdflatex; use eps in DVI mode
                                                                % TeX will automatically convert eps --> pdf in pdflatex                
\usepackage{amssymb}

\usepackage[spanish]{babel}                        % Permite que partes automáticas del documento aparezcan en castellano.
\usepackage[utf8]{inputenc}                        % Permite escribir tildes y otros caracteres directamente en el .tex
\usepackage[T1]{fontenc}                                % Asegura que el documento resultante use caracteres de una fuente apropiada.

\usepackage{hyperref}                                % Permite poner urls y links dentro del documento

\title{Ejercicios de Programación - Sebesta}
\author{Lenguajes de Programación - ESPOL}

%\date{}                                                        % Activate to display a given date or no date

\begin{document}
\maketitle

\section{Introducción}
Las respuestas propuestas en este repositorio son producto del trabajo de los estudiantes de la materia ``Lenguajes de Programación'' de la ESPOL, correspondientes a las preguntas del libro de Robert Sebesta, Concepts of Programming Languages.

\section{Preguntas y Respuestas}

\subsection{Capítulo 5: Nombres, Enlaces y Alcances.}
\input{c5p4}
%\subsubsection{Pregunta 4: Escriba un script en Python con subprogramas triplemente anidados, donde cada subprograma referencie a variables que han sido definidas en otros niveles de la jerarquía}

Escriba su respuesta con claridad. En las secciones de código utilice listings.

Capitulo 5 pregunta 5
En caso de querer llamar en nuestro programa números pseudoaleatorios, que han sido generados por un subprograma se necesita del history-sensitivy para almacenar la última iteración de la variable localmente. 
Ejemplo de un subprograma generando numeros pseudoaleatorios:

 public void generarSerieDeAleatorios () \{  \\

        Random numAleatorio;

        numAleatorio = new Random ();

        for (int i=0; i < serieAleatoria.size(); i++)  \{

        serieAleatoria.set(i, numAleatorio.nextInt(1000) );

         \}

\subsection{Capítulo 9: SubProgramas.}
\subsubsection{Pregunta 4:Escribir en C\# un programa que pase por referencia un literal a un subprograma, que intenta cambiar el parametro. Dado la filosofia total de C\#, explique los resultados.}
\lstset{language = C}  %Seteo para poner Codigo de Lenguaje C
Código creado para la explicación en C\#.
\begin{lstlisting}[frame = single] %Comienzo del Código

 static void Main(string[] args)
        {
            String a;
            int b= -1;
            Program p = new Program();
            System.Console.Write("Ingrese numero:\n");
            a = Console.ReadLine();
            System.Console.Write("El numero es: " +a);
            System.Console.Write("\n");
            b = Convert.ToInt32(a);
            p.fun(ref b);
            System.Console.Write("El numero enviado a la funcion por referencia es: "+b);
            System.Console.Read();
            //System.Console.ReadKey();
        }

        public void fun (ref int x){
            x = x + 5;  
        }
\end{lstlisting}
OUTPUT:\\
	Ingrese numero: 6\\
	El numero es: 6\\
	El numero enviado a la funcion por referencia es:11\\
	El lenguague C\# recibe por defecto cuando se ingresa por teclado un Strign a este valor hay que convertirlo a un entero para enviarlo a nuestra funcion con la que queremos mostrar lo que deseamos.\\
 Lo que queremos mostrar es el cambio que se produce al valor de una variable siendo enviada a un subrprograma por referencia.\\


%\subsubsection{Pregunta 6: Escriba programas de prueba en C++, Java, y C\# para determinar el alcance de una variable declarada en una sentencia for. Específicamente, el código debe determinar si esta variable es visible después del cuerpo de la sentencia for.}

\lstinputlisting[language=C++, frame=single,caption=Lenguaje C++]{c5p6.cpp}

\lstinputlisting[language=Java,frame=single, caption=Lenguaje Java]{c5p6.java}

\lstinputlisting[language=C,frame=single, caption= Lenguaje C \#]{c5p6.cs}

Podemos observar que en todos los tres lenguajes, el declarar una variable dentro del for y luego intentar accederla fuera del bloque for, nos genera un error. Esto se debe a que la variable tiene alcance local dentro del bloque for, y solo puede ser accedida dentro del mismo.

%\input{c5p7}
Capitulo 5 pregunta 7
Supongamos que el siguiente programa de JavaScript fue interpretado utilizando las reglas de static-scoping.
¿Que valor de X se muestra en la funcion sub1? 
Respuesta: El valor de X que muestra la funcion sub1 es 5.
Bajo las reglas  de Dynamic-scoping, ¿Que valor de X muestra la funcion sub1?
Respuesta: El valor de X que muestra la funcion sub1 es 10.

var x;
function sub1() {
document.write("x = " + x + "<br />");
}
function sub2() {
var x;
x = 10;
sub1();
}
x = 5;
sub2();

Static-scoping
 Las  declaraciones locales de un funcion sólo incluyen a los presentes en la función y no a los que puedan estar presentes en funciones anidadas  dentro de esa función.
 Si se declara una variable con el mismo nombre de una variable fuera de la funcion,estas variables no se relacionan, son totalmente distintas.

Dynamic-scoping
 El uso de esta norma de alcance, primero buscamos una definición local de una variable. Si no lo encuentra, se busca en la pila de llamadas para una definición.




% Continuar con los siguientes capítulos y ejercicios:
% Ch6: 1, 2, 7
% Ch7: 1 - 6, 9
% Ch8: 3, 4, 5
% Ch9: 1, 5
% Recuerden que todos corresponden a las secciones de "Programming Exercises".

\end{document}
