\documentclass[12pt,oneside]{article}
\usepackage{geometry}                                % See geometry.pdf to learn the layout options. There are lots.
\geometry{a4paper}                                           % ... or a4paper or a5paper or ... 
%\geometry{landscape}                                % Activate for for rotated page geometry
%\usepackage[parfill]{parskip}                    % Activate to begin paragraphs with an empty line rather than an indent
\usepackage{graphicx}                                % Use pdf, png, jpg, or epsß with pdflatex; use eps in DVI mode
                                                                % TeX will automatically convert eps --> pdf in pdflatex                
\usepackage{amssymb}

\usepackage[spanish]{babel}                        % Permite que partes automáticas del documento aparezcan en castellano.
\usepackage[utf8]{inputenc}                        % Permite escribir tildes y otros caracteres directamente en el .tex
\usepackage[T1]{fontenc}                                % Asegura que el documento resultante use caracteres de una fuente apropiada.

\usepackage{hyperref}                                % Permite poner urls y links dentro del documento

\title{Ejercicios de Programación - Sebesta}
\author{Lenguajes de Programación - ESPOL}
%\date{}                                                        % Activate to display a given date or no date

\begin{document}
\maketitle

\section{Introducción}
Las respuestas propuestas en este repositorio son producto del trabajo de los estudiantes de la materia ``Lenguajes de Programación'' de la ESPOL, correspondientes a las preguntas del libro de Robert Sebesta, Concepts of Programming Languages.

\section{Preguntas y Respuestas}

\subsection{Capítulo 5: Nombres, Enlaces y Alcances}
\input{c5p4}
Están estrechamente relacionados porque las variables implícitas de pila dinámica están obligadas
a un almacenamiento dinámico cuando se asignan valores mediante un enlace de tipo dinámico que es el que permite asignar valores de cualquier tipo a cualquier variable.

\subsubsection{Pregunta 4: Escriba un script en Python con subprogramas triplemente anidados, donde cada subprograma referencie a variables que han sido definidas en otros niveles de la jerarquía}

Escriba su respuesta con claridad. En las secciones de código utilice listings.

Capitulo 5 pregunta 5
En caso de querer llamar en nuestro programa números pseudoaleatorios, que han sido generados por un subprograma se necesita del history-sensitivy para almacenar la última iteración de la variable localmente. 
Ejemplo de un subprograma generando numeros pseudoaleatorios:

 public void generarSerieDeAleatorios () \{  \\

        Random numAleatorio;

        numAleatorio = new Random ();

        for (int i=0; i < serieAleatoria.size(); i++)  \{

        serieAleatoria.set(i, numAleatorio.nextInt(1000) );

         \}
\subsubsection{Pregunta 6: Escriba programas de prueba en C++, Java, y C\# para determinar el alcance de una variable declarada en una sentencia for. Específicamente, el código debe determinar si esta variable es visible después del cuerpo de la sentencia for.}

\lstinputlisting[language=C++, frame=single,caption=Lenguaje C++]{c5p6.cpp}

\lstinputlisting[language=Java,frame=single, caption=Lenguaje Java]{c5p6.java}

\lstinputlisting[language=C,frame=single, caption= Lenguaje C \#]{c5p6.cs}

Podemos observar que en todos los tres lenguajes, el declarar una variable dentro del for y luego intentar accederla fuera del bloque for, nos genera un error. Esto se debe a que la variable tiene alcance local dentro del bloque for, y solo puede ser accedida dentro del mismo.

a.sub2 y sub3--Se determina en el momento que se ejecuta el codigo en el subrograma 2 y 3 tomara la primera 
asignacion a la variable x dada en el main(variable global).

b.sub 1--  La union de la variable se determina en el momento que se ejecuta el programa, por lo tanto toma el 
valor del ultimo subprograma que asigno valor a la variable.

\input{c5p7}
reglas de alcance estático-- 5 variable global
reglas de alcance dinámico--10 variable local

\input{c9p1}
Ventaja
El usuario tiene mayor flexibilidad.El lenguaje que tiene soporte a las funciones o tipos de datos definidas por el usuario, permiten que el utilice funciones definidas para trabajarlas con sus objetos en base a sus necesidades.

Overloading--- Legibilidad y flexibilidad

 \\
Desventaja
El usuario podria hacer construcciones de una manera incorrecta y caer en problemas de fiabilidad.

Overloading---Operaciones confusas y no obtener el resultado esperado.

Ejemplos overloading:

Phyton  \\
def suma(a,b):  \\
    resultado=a+b  \\
    return resultado  \\ \\

def suma(a,b,c)  \\
     resultado=a+b+c  \\
     return resultado  \\

C++  \\
class imprimir  \\
\{  \\
public:  \\
void print (char * c)\{  \\
 cout << “Imprime frase:”<<c<<endl;  \\
\}  \\
void print (int i)\{  \\
 cout << “Imprime entero:”<<i<<endl;  \\
 \}  \\
\}  \\



% Continuar con los siguientes capítulos y ejercicios:
% Ch6: 1, 2, 7
% Ch7: 1 - 6, 9
% Ch8: 3, 4, 5
% Ch9: 1, 5
% Recuerden que todos corresponden a las secciones de "Programming Exercises".

\end{document}
