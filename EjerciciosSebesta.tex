\documentclass[12pt,oneside]{article}
\usepackage{geometry}                                % See geometry.pdf to learn the layout options. There are lots.
\geometry{a4paper}                                           % ... or a4paper or a5paper or ... 
%\geometry{landscape}                                % Activate for for rotated page geometry
%\usepackage[parfill]{parskip}                    % Activate to begin paragraphs with an empty line rather than an indent
\usepackage{graphicx}                                % Use pdf, png, jpg, or epsß with pdflatex; use eps in DVI mode
                                                                % TeX will automatically convert eps --> pdf in pdflatex                
\usepackage{amssymb}

\usepackage[spanish]{babel}                        % Permite que partes automáticas del documento aparezcan en castellano.
\usepackage[utf8]{inputenc}                        % Permite escribir tildes y otros caracteres directamente en el .tex
\usepackage[T1]{fontenc}                                % Asegura que el documento resultante use caracteres de una fuente apropiada.

\usepackage{hyperref}                                % Permite poner urls y links dentro del documento

\title{Ejercicios de Programación - Sebesta}
\author{Lenguajes de Programación - ESPOL}
%\date{}                                                        % Activate to display a given date or no date

\begin{document}
\maketitle

\section{Introducción}
Las respuestas propuestas en este repositorio son producto del trabajo de los estudiantes de la materia ``Lenguajes de Programación'' de la ESPOL, correspondientes a las preguntas del libro de Robert Sebesta, Concepts of Programming Languages.

\section{Preguntas y Respuestas}

\subsection{Capítulo 5: Nombres, Enlaces y Alcances}
\subsubsection{Pregunta 4: Escriba un script en Python con subprogramas triplemente anidados, donde cada subprograma referencie a variables que han sido definidas en otros niveles de la jerarquía}

Escriba su respuesta con claridad. En las secciones de código utilice listings.

\subsection{Pregunta 5: Escriba una función en C que incluya las siguienes líneas de código: \newline
x = 21; \newline
int x; \newline
x = 42; \newline
Corra el programa y explique los resultados. Reescriba el mismo código en C++ y Java.}

Ninguno de los siguientes dos bloques de código compilan en C. En el listing 1, la variable 'x' aún no esta declarada al usarse en la primera asignación.
En el listing 2, hay una nueva declaración de la variable 'x', generando una contradicción.

\lstset{language=C}          % Set your language (you can change the language for each code-block optionally)

\begin{lstlisting}[caption= Pregunta 5 Capítulo 5, label=amb, frame=single]  % Start your code-block
  
#include<stdio.h>

main()
{
	x = 21;
	int x;
	x = 42;

	printf("%d", x);
	getch();

}
\end{lstlisting}
\begin{lstlisting}[caption= Pregunta 5 Capítulo 5, label=amb, frame=single]  % Start your code-block
  
#include<stdio.h>

main()
{
	int x;
	x = 21;
	int x;
	x = 42;

	printf("%d", x);
	getch();

}
\end{lstlisting}

En el listing 3, el programa imprime el valor de 21. Esto se debe a que la variable global (‘x’) es modificada al llamarse a la función ‘foo’. Dentro de esa función se crea, asimismo, una variable ‘x’ que esta solo dentro del scope de la función ‘foo’. Dentro del scope del ‘main’ solo está la variable ‘x’ global y no la ‘x’ de la función ‘foo’, por ende imprime ese valor que se le fue asignado al llamar a ‘foo’.
En el listing 4, hay una nueva declaración de la variable 'x', generando una contradicción.

\lstset{language=C++}          % Set your language (you can change the language for each code-block optionally)

\begin{lstlisting}[caption= Pregunta 5 Capítulo 5, label=amb, frame=single]  % Start your code-block
  
#include <iostream>
using namespace std;

int x ;

void foo () { 
    x =  21 ; 
    int x ; 
    x =  42 ;   
}

int main() {
   foo();
   printf("%d",x);
   return 0;
}
\end{lstlisting}

\begin{lstlisting}[caption= Pregunta 5 Capítulo 5, label=amb, frame=single]  % Start your code-block
  
#include <iostream>
using namespace std;

int main(void) {
	int x;
	x = 21;
	int x;
	x = 42;
	cout << x << endl;
	system("pause");
	return 0;
}
\end{lstlisting}

En el listing 5, lo primero que se realiza es la modificación a la variable global (x) a 21. Luego, se crea otra variable x (dentro de la función) y se le asigna a esta el valor de 42, muy aparte de la variable global, cuyo valor seguirá siendo 21. En el Main, el scope solo incluye a la variable global x, ya que la otra existe solo dentro del scope de la función. Por ende, solo se mostrara la variable global, con el valor que se le asignó al llamar a la función.
En el listing 6, hay una nueva declaración de la variable 'x', generando una contradicción.

\lstset{language=Java}          % Set your language (you can change the language for each code-block optionally)

\begin{lstlisting}[caption= Pregunta 5 Capítulo 5, label=amb, frame=single]  % Start your code-block
  
public class Ejercicio5
{
    static int x;
    public static void main( String[] args )
    {
        Ejercicio5 ej5 = new Ejercicio5();
        ej5.ejercicio5();
        System.out.println(x);
    }
    public void ejercicio5(){
        x = 21;
        int x;
        x = 42;
    }
}
\end{lstlisting}

\begin{lstlisting}[caption= Pregunta 5 Capítulo 5, label=amb, frame=single]  % Start your code-block
  
public static void main(String[] args) {
        int x;
        x = 21;
        int x;
        x = 42;
        
        System.out.println(x);
    }
\end{lstlisting}

Capitulo 5 pregunta 5
En caso de querer llamar en nuestro programa números pseudoaleatorios, que han sido generados por un subprograma se necesita del history-sensitivy para almacenar la última iteración de la variable localmente. 
Ejemplo de un subprograma generando numeros pseudoaleatorios:

 public void generarSerieDeAleatorios () \{  \\

        Random numAleatorio;

        numAleatorio = new Random ();

        for (int i=0; i < serieAleatoria.size(); i++)  \{

        serieAleatoria.set(i, numAleatorio.nextInt(1000) );

         \}
%\subsubsection{C5 Pregunta6: Sentencia for alcance de la variabe declarada en ella en los tres lenguajes acontinuacion explicados }

\begin{verbatim}

JAVA
import java.io.*;
 
public class c5p6{
    public static void main(String args[]){
 	for(int i = 0; i< 10 ; i++)
 	    {
 		System.out.println(i);
 	    }
 	System.out.println(i);
     }
 }

C++
#include <iostream>
using namespace std;
 
int main(){
   for (int i = 0 ; i<10; i++){
     cout << i;
   }
   cout << i; 
   return 0;
 }

C#
using System;
 
 class c5p6
 {
   static void Main()
   {
     for(int i = 0 ; i < 10 ; i++)
       {
 	Console.WriteLine(i);
       }
     Console.WriteLine(i);
   }
 
 }

\end{verbatim}
En los tres lenguajes, el declarar una variable dentro del for\\
y luego intentar accederla fuera del bloque for, nos genera \\
un error. La variable tiene alcance local dentro del bloque for, y \\
solo puede ser accedida dentro del mismo.


%\subsubsection{Pregunta 7: Tres funciones en C donde se declare un arreglo de forma\\
estatica, otra stack y la ultima declaracion como heap}

\begin{verbatim}
STACK 
int main (int argc, char *argv[])
{
 
  int arreglo[3000];
  int i;
  srand(time(NULL));

  for(i=0;i<2999;i++)
  arreglo[i]=1+rand()%100;

  for(i=0;i<2999;i++){
  int doble;
  doble=arreglo[i]*2;
  printf("El doble del numero aleatorio en la posicion %d manejado por pila es: %d\n",i,doble);
 
  }
  return 0;
}

HEAP
int main (int argc, char *argv[])
{
 
  int arreglo[3000]; 
  int i;
  int *p;
  srand(time(NULL));
  


  for(i=0;i<2999;i++){
  p= (int *)malloc(3000*sizeof(int));
  arreglo[i]=1+rand()%100;
  *p=arreglo[i];
  }
 

  for(i=0;i<2999;i++){
  int doble;
  p= (int *)malloc(3000*sizeof(int));
  *p=arreglo[i];
  doble=*p*2;
  printf("El doble del numero aleatorio en la posicion %d manejado por heap es: %d\n",i,doble);
  
  }

  free(p);
  return 0;
}

STATIC

int main (int argc, char *argv[])
{
 
  static int arreglo[3000];
  int i;
  srand(time(NULL));

  for(i=0;i<2999;i++)
  arreglo[i]=1+rand()%100;

  for(i=0;i<2999;i++){
  int doble;
  doble=arreglo[i]*2;
  printf("El doble del numero aleatorio en la posicion %d con arreglo estatico es: %d\n",i,doble);
 
  }
  return 0;
}

\end{verbatim}
Stack tiene un aceso mas rapido, el espacio es manejado por el CPU es limitado y no puede ser redefinido.\\
En el caso del heap nosotros necesitamos manejar la memoria, acceso mas lento y no es limitado\\
Static declaracion unica de una variable que mantiene su dimension a lo largo del tiempo de vida del programa.


Capitulo 5 pregunta 7
Supongamos que el siguiente programa de JavaScript fue interpretado utilizando las reglas de static-scoping.
¿Que valor de X se muestra en la funcion sub1? 
Respuesta: El valor de X que muestra la funcion sub1 es 5.
Bajo las reglas  de Dynamic-scoping, ¿Que valor de X muestra la funcion sub1?
Respuesta: El valor de X que muestra la funcion sub1 es 10.

var x;
function sub1() {
document.write("x = " + x + "<br />");
}
function sub2() {
var x;
x = 10;
sub1();
}
x = 5;
sub2();

Static-scoping
 Las  declaraciones locales de un funcion sólo incluyen a los presentes en la función y no a los que puedan estar presentes en funciones anidadas  dentro de esa función.
 Si se declara una variable con el mismo nombre de una variable fuera de la funcion,estas variables no se relacionan, son totalmente distintas.

Dynamic-scoping
 El uso de esta norma de alcance, primero buscamos una definición local de una variable. Si no lo encuentra, se busca en la pila de llamadas para una definición.




% Continuar con los siguientes capítulos y ejercicios:
% Ch6: 1, 2, 7
% Ch7: 1 - 6, 9
% Ch8: 3, 4, 5
% Ch9: 1, 5
% Recuerden que todos corresponden a las secciones de "Programming Exercises".

\end{document}
