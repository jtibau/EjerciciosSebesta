\documentclass[12pt,oneside]{article}
\usepackage{geometry}                                % See geometry.pdf to learn the layout options. There are lots.
\usepackage{listings}				% Permite utilizar lenguajes de programacion dentro de latex
\geometry{a4paper}                                           % ... or a4paper or a5paper or ... 
%\geometry{landscape}                                % Activate for for rotated page geometry
%\usepackage[parfill]{parskip}                    % Activate to begin paragraphs with an empty line rather than an indent
\usepackage{graphicx}                                % Use pdf, png, jpg, or epsß with pdflatex; use eps in DVI mode
                                                                % TeX will automatically convert eps --> pdf in pdflatex                
\usepackage{amssymb}

\usepackage[spanish]{babel}                        % Permite que partes automáticas del documento aparezcan en castellano.
\usepackage[utf8]{inputenc}                        % Permite escribir tildes y otros caracteres directamente en el .tex
\usepackage[T1]{fontenc}                                % Asegura que el documento resultante use caracteres de una fuente apropiada.

\usepackage{hyperref}                                % Permite poner urls y links dentro del documento

\title{Ejercicios de Programación - Sebesta}
\author{Lenguajes de Programación - ESPOL}
%\date{}                                                        % Activate to display a given date or no date

\begin{document}
\maketitle

\section{Introducción}
Las respuestas propuestas en este repositorio son producto del trabajo de los estudiantes de la materia ``Lenguajes de Programación'' de la ESPOL, correspondientes a las preguntas del libro de Robert Sebesta, Concepts of Programming Languages.

\section{Preguntas y Respuestas}

\subsection{Capítulo 5: Nombres, Enlaces y Alcances.}
\input{c5p4}
%\subsubsection{Pregunta 4: Escriba un script en Python con subprogramas triplemente anidados, donde cada subprograma referencie a variables que han sido definidas en otros niveles de la jerarquía}

Escriba su respuesta con claridad. En las secciones de código utilice listings.

Capitulo 5 pregunta 5
En caso de querer llamar en nuestro programa números pseudoaleatorios, que han sido generados por un subprograma se necesita del history-sensitivy para almacenar la última iteración de la variable localmente. 
Ejemplo de un subprograma generando numeros pseudoaleatorios:

 public void generarSerieDeAleatorios () \{  \\

        Random numAleatorio;

        numAleatorio = new Random ();

        for (int i=0; i < serieAleatoria.size(); i++)  \{

        serieAleatoria.set(i, numAleatorio.nextInt(1000) );

         \}
\subsection{Capítulo 7: Expresiones y Asignaciones de sentencias.}
\subsubsection{Pregunta 2: Reescribir el programa del ejercicio 1 en C++,en Java,en C\# y ejecutarlos y comparar los resultados.}
\lstset{language = C++}  %Seteo para poner Codigo de Lenguaje C++
Código traducido de Lenguaje C a lenguaje C++
\begin{lstlisting}[frame = single] %Comienzo del Código

#include <stdio.h>
int fun(int *k);
int main() {
    int i = 10, j = 10, sum1, sum2;
    sum1 = (i / 2) + fun(&i);
    sum2 = fun(&j) + (j / 2);
    printf("sum1 =  %i  \n ",sum1);
    printf("sum2 =  %i ",sum2);
    return 0;
}

int fun(int *k) {
    *k += 4;
    return 3 * (*k) - 1;
}
\end{lstlisting}
OUTPUT:\\
	sum1 =  46\\
	sum2 =  48\\
En C++ a diferencia de C debemos declara el prototipo de la función antes del main, por los resultados obtenidos la regla de asociatividad del lenguaje en C++ es igual a la del Lenguaje C, es decir la Asociatividad es de izquierda a derecha.\\\\
\lstset{language = C}  %Seteo para poner Codigo de Lenguaje C
Código traducido de Lenguaje C++ a lenguaje C\#
\begin{lstlisting}[frame = single] %Comienzo del Código

class Program
    {
        static void Main(string[] args)
        {
            Program p = new Program();
            int i = 10,j=10,sum1,sum2;
            sum1 = (i / 2) + p.fun(ref i);
            sum2 = p.fun(ref j) + (j / 2);
            System.Console.WriteLine(" "+sum1);
            System.Console.WriteLine(" "+sum2);
            Console.Read();
        }
        public int fun(ref int k)
        {
            k = 4 + k;
            return 3*(k) - 1;
        }
    }
\end{lstlisting}
OUTPUT:\\
	sum1 =  46\\
	sum2 =  48\\
El órden de las reglas de asociatividad del lenguaje C\# es el mismo que tiene c++ y c que es el mismo de izquierda a derecha aquí el puntero se reemplaza con la palabra clave "reference" produce un argumento que se pasa por referencia, no por valor.\\\\
\lstset{language = java}  %Seteo para poner Codigo de Lenguaje C#
Código traducido de Lenguaje C a JAVA
\begin{lstlisting}[frame = single] %Comienzo del Código

public class Prueba {
    public static void main(String[] args) {
        Prueba p = new Prueba();
        int i = 10, j = 10, sum1, sum2;
        sum1 = (i/2) + p.fun(i);
        sum2 = p.fun(j) + (j/2);
        System.out.println("Valor 1: "+sum1);
        System.out.println("Valor 2: "+sum2);
    }

    public int fun(int k){
        k += 4;
        return (3 * (k) - 1);
    }
}
\end{lstlisting}
OUTPUT:\\
	sum1 =  46\\
	sum2 =  46\\
Java tiene un tipo de apuntadores o, mejor dicho, referencias. Y no tiene operadores que tenga que poner explícitamente como \*,\^ o \&\&; y no pueden ser manipularlos con operaciones aritméticas o de otro tipo como en C.
Entonces el orden de las reglas de asociatividad del lenguaje en Java.\\

%\subsubsection{Pregunta 6: Escriba programas de prueba en C++, Java, y C\# para determinar el alcance de una variable declarada en una sentencia for. Específicamente, el código debe determinar si esta variable es visible después del cuerpo de la sentencia for.}

\lstinputlisting[language=C++, frame=single,caption=Lenguaje C++]{c5p6.cpp}

\lstinputlisting[language=Java,frame=single, caption=Lenguaje Java]{c5p6.java}

\lstinputlisting[language=C,frame=single, caption= Lenguaje C \#]{c5p6.cs}

Podemos observar que en todos los tres lenguajes, el declarar una variable dentro del for y luego intentar accederla fuera del bloque for, nos genera un error. Esto se debe a que la variable tiene alcance local dentro del bloque for, y solo puede ser accedida dentro del mismo.

%\input{c5p7}

% Continuar con los siguientes capítulos y ejercicios:
% Ch6: 1, 2, 7
% Ch7: 1 - 6, 9
% Ch8: 3, 4, 5
% Ch9: 1, 5
% Recuerden que todos corresponden a las secciones de "Programming Exercises".

\end{document}
