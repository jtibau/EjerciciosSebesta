
\documentclass{article}
\usepackage{listings}
\usepackage{graphicx} % support the \includegraphics command and options

% \usepackage[parfill]{parskip} % Activate to begin paragraphs with an empty line rather than an indent

%%% PACKAGES
\usepackage{booktabs} % for much better looking tables
\usepackage{array} % for better arrays (eg matrices) in maths
\usepackage{paralist} % very flexible & customisable lists (eg. enumerate/itemize, etc.)
\usepackage{verbatim} % adds environment for commenting out blocks of text & for better verbatim
\usepackage{subfig} % make it possible to include more than one captioned figure/table in a single float
\usepackage{setspace} %paquete para interlineado
\usepackage{graphicx} %para insertar graficos
\usepackage{parskip} % npi de q es
\usepackage{color} %colores
\usepackage{float}             % Include the listings-package
\begin{document}
\lstset{language=JAva}          % Set your language (you can change the language for each code-block optionally)
\title{Sebesta Chapter 8 - Ex 5}
\begin{lstlisting}[frame=single]  % Start your code-block
/**
 * Fausto Mora - Cap 8 ex 5
 */
package sebesta;

/**
 *
 * @author Lost Legion
 */
public class chapterEightExFive {
    
    public static void main(String args[]){
        boolean encontrado =false;
        int n=5; //suponemos la dimesion 5
        int[][] x = new int[n][n];
        
        for(int i=0;i<=n;i++){
            int acum=0;
            for(int j=0;j<=n;j++){
                if(x[i][j]==0){
                    acum++;
                }
                if(acum==n && !encontrado){
                    System.out.println("Encontrado¡¡¡");
                    encontrado=true;
                }
            }      
        }
        
    }
}
}
\end{lstlisting}
Aunque el codigo presentado en el programa es mas corto, con seguridad el mismo ejercicio transcrito a lenguaje java presenta mayor legibilidad
\end{document}