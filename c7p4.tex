\subsubsection{C7 Pregunta 4}
\begin{lstlisting}
 public class JavaApplication2 {
     final static int num=5;
     static int a=5;
    
     public static void main(String[] args) {          
           a = fun1()+a; 
           System.out.println(a); 
     }
 
     static int fun1() {
             a = 17;
             return 3;
        }
 
 }
 
 \end{lstlisting}
 OUTPUT\\
 20\\
 Como podemos ver en el main en la 1era linea, fun1() retorna 3 y actualiza la variable global 'a'=17 y luego suma 17+3 asignando 20 a 'a'. Esto se debe a que en JAVA el operador '+' tiene asociatividad desde la izquierda.\\
 En cambio si fuese:  a = a + fun1();. En este caso primero 'a'=5 + fun1() que retorne 3; dando una asignación de 8 a la variable a.