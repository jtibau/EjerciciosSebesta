
\documentclass{article}
\usepackage{listings}
\usepackage{graphicx} % support the \includegraphics command and options

% \usepackage[parfill]{parskip} % Activate to begin paragraphs with an empty line rather than an indent

%%% PACKAGES
\usepackage{booktabs} % for much better looking tables
\usepackage{array} % for better arrays (eg matrices) in maths
\usepackage{paralist} % very flexible & customisable lists (eg. enumerate/itemize, etc.)
\usepackage{verbatim} % adds environment for commenting out blocks of text & for better verbatim
\usepackage{subfig} % make it possible to include more than one captioned figure/table in a single float
\usepackage{setspace} %paquete para interlineado
\usepackage{graphicx} %para insertar graficos
\usepackage{parskip} % npi de q es
\usepackage{color} %colores
\usepackage{float}             % Include the listings-package
\begin{document}
\lstset{language=JAva}          % Set your language (you can change the language for each code-block optionally)
\title{Sebesta Chapter 8 - Ex 5}
\begin{lstlisting}[frame=single]  % Start your code-block
/**
 * Fausto Mora - Cap 8 ex 5
 */
package sebesta;

/**
 *
 * @author Lost Legion
 */
public class chapterSevenExNine {
    
    public void floatOperations(){
        double[] op = new double[200];
        
        for(int i=0; i<200;i++){
            op[i]=Math.random();
        }
        
        for(int j=0;j<200;j++){
            op[j] = op[j] + Math.random();
            op[j] = op[j] - Math.random();
            op[j] = op[j] * Math.random();
        }
    
    }
    
    public void intOperations(){
        int[] op = new int[200];
        
        for(int i=0; i<200;i++){
            op[i]=(int)Math.random();
        }
        
        for(int j=0;j<200;j++){
            op[j] = op[j] + (int)Math.random();
            op[j] = op[j] - (int)Math.random();
            op[j] = op[j] * (int)Math.random();
        }
    }
    
    public static void main(String args[]){
        long startTimeF = System.nanoTime();
        new  chapterSevenExNine().floatOperations();
        long endTimeF = System.nanoTime();
        long durationF = endTimeF - startTimeF;
        System.out.println("Tiempo de procesos de punto flotante: " + durationF);


        long startTimeI = System.nanoTime();
        new chapterSevenExNine().intOperations();
        long endTimeI = System.nanoTime();
        long durationI = endTimeI - startTimeI;
        System.out.println("Tiempor de procesos de tipo enteros: " + durationI);
    }

\end{lstlisting}
El tiempo medio de las operaciones enteros fueron al rededor de : 25000mils
\\ Mientras el tiempo de operaciones de punto float casi tres veces mayores, al rededor de: 74000mils
\end{document}