\subsubsection{C6 Pregunta 1}
\begin{lstlisting}
int main (int argc, char *argv[])
{
	int i,div_int;
	float div_float,j=11.00;
	int arreglo[3];
	arreglo[2]=10;
	div_int=arreglo[2]/2;
	div_float=j/2;
	if(div_int!=div_float){
		printf("%i diferente %.2f",div_int,div_float);
		 Sleep(1000);
	}if(div_int==div_float){
	     printf("%i igual %.2f",div_int,div_float);
		 Sleep(1000);
	}
	
  return 0;
}
 \end{lstlisting}
La compatibilidad de tipos en C es muy variada, si comparamos como en el primer \\
if es claro que apesar de que los dos son numeros cinco los decimales de el segundo\\
valor hacen que la igualdad no se cumpla en caso de haber salido un 5.00 la igualdad \\
se cumplia ignorando que sean de diferentes tipos.Esto hace que cuando comparamos en \\
c el se encargue de algunas conversiones para llevar a caboo comparaciones en el \\
programa, en caso de que cambiemos a int el divfloat nos advertira de la perdida de datos \\
pero no nos generara error alguno.
  