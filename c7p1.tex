\subsubsection{Pregunta 1:Ejecute el código dado en el Problema 13(Set de Problemas) en un sistema que soporte C para determinar los valores de sum1 y sum2. Explique los resultados.}
\lstset{language = C}  %Seteo para poner Codigo de Lenguaje C
Código dado en el Problema 13(Set de Problemas)
\begin{lstlisting}[frame = single] %Comienzo del Código

int fun(int *k){
	*k+=4; /* Incrementa el valor de x + 4*/
	return 3 * (*k) -1 /* Retorna */
}

void main(){
	int i=10, j=10,sum1,sum2;
	sum1 =(i/2) + fun(&i);
	sum2 = fun(&j) + (j/2);
}
\end{lstlisting}
OUTPUT:\\
	sum1 = 46\\
	sum2 = 48\\
	Estos resultados se da por las Reglas de Asociatividad del lenguaje C, 
	que es la asociatividad de izquierda a derecha, esto significa que e operador de la izquierda se evaluara primero.\\
	
	Para sum2, la dirección de memoria cambia debido a las operaciones que se realiza
	con la refercencia a la variable fun, por esta razón hay diferentes resultados.\\


