
\documentclass{article}
\usepackage{listings}
\usepackage{graphicx} % support the \includegraphics command and options

% \usepackage[parfill]{parskip} % Activate to begin paragraphs with an empty line rather than an indent

%%% PACKAGES
\usepackage{booktabs} % for much better looking tables
\usepackage{array} % for better arrays (eg matrices) in maths
\usepackage{paralist} % very flexible & customisable lists (eg. enumerate/itemize, etc.)
\usepackage{verbatim} % adds environment for commenting out blocks of text & for better verbatim
\usepackage{subfig} % make it possible to include more than one captioned figure/table in a single float
\usepackage{setspace} %paquete para interlineado
\usepackage{graphicx} %para insertar graficos
\usepackage{parskip} % npi de q es
\usepackage{color} %colores
\usepackage{float}             % Include the listings-package
\begin{document}
\lstset{language=JAva}          % Set your language (you can change the language for each code-block optionally)
\title{Sebesta Chapter 8 - Ex 5}
\begin{lstlisting}[frame=single]  % Start your code-block
/**
 * Fausto Mora - Cap 8 ex 3
 */
package sebesta;

/**
 *
 * @author Lost Legion
 */
public class chapterEightExThree {
    //realizaremos el ejercicio 3 del cap 8 la opcion C
    //es decir en lenguaje JAVA
    
    public static void main(String args[]){
        int j=0; int k=0;
        
        switch(k){
            case 1: j= 2 * k -1;
            case 2: j = 2 * k - 1;
            case 3: j = 3 * k + 1;
            case 4: j = 4 * k - 1;
            case 5: j = 3 * k + 1;
            case 6: j = k - 2;
            case 7: j = k - 2;
            case 8: j = k - 2;
            default: System.out.println("Fuera de rango¡¡¡");
                
        
        }
        
    
    }
}
\end{lstlisting}

\end{document}